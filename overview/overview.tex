\documentclass{scrartcl}

\title{Open System-on-Chip Debug}
\author{Stefan Wallentowitz, ...}
\date{}

\begin{document}

\maketitle

\begin{abstract}

  With the ever increasing number of open source system-on-chip
  design, we see a big benefit in a unified system-on-chip debug
  infrastructure. On the one hand it is fair to say that developers
  usually see debug as a inevitable must that does not add some fancy
  new feature or increases performance, hence a toolbox of available
  building blocks can easily help them. On the other hand a unified
  infrastructure with shared interfaces eases the interoperability of
  tools and hardware.

  The goal of the Open System-on-Chip Debug infrastructure is
  therefore to provide a repository of building blocks, glue
  infrastructure and host software libraries for SoC debugging. We
  therefore aim at supporting support for both run-control debugging
  and trace debugging. While run-control debugging is well-known and
  often used in the shape of JTAG-based debugging cables,
  trace-debugging is the observation of a system-on-chip with trace
  events and it gains increasing significance with multicore
  system-on-chip due to their parallelism.

  In this article we give an overview of the the Open SoC Debug
  project and the different layers and components both on the chip and
  on the host the we plan in this project.
\end{abstract}

\section*{Introduction}

\section*{Basic Debug Infrastructure}

\subsection*{Physical Interface}

\subsection*{Transport Layer}

\subsection*{Switching Layer}

\section*{Basic Host Interface}

\section*{Building Blocks}

\subsection*{System Description Module}

\subsection*{JTAG Endpoint Module}

\subsection*{Core Debug Modules}

\subsection*{Core Trace Modules}

\subsection*{Memory Access Modules}

\subsection*{Debug Processor Modules}

\end{document}